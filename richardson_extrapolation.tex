\documentclass[]{article}
% Time-stamp: <2014-01-17 19:36:04 (jonah)>

% packages
\usepackage{fullpage}
\usepackage{amsmath}
\usepackage{amssymb}
\usepackage{latexsym}
\usepackage{graphicx}
\usepackage{mathrsfs}
\usepackage{verbatim}
\usepackage{braket}
\usepackage{listings}
\usepackage{hyperref}

% Preamble
\hypersetup{pdftitle={background}}
\author{Jonah Miller\\
  \href{mailto:jonah.maxwell.miller@gmail.com}{jonah.maxwell.miller@gmail.com}}
\title{Richardson Extrapolation}

% Macros
\newcommand{\R}{\mathbb{R}}
\newcommand{\eval}{\biggr\rvert} %evaluated at

\begin{document}
\maketitle

Suppose that we are numerically solving a differential equation by
finite differences. The true answer is a function
\begin{equation}
  \label{eq:f^*(x)}
  f^*(x)
\end{equation}
which we will often denote $f^*$ for brevity. A numerical solution to
the problem is modified by the lattice spacing $h$ and a factor
$\alpha(x)$ which is a function of the problem itself:
\begin{equation}
  \label{eq:f(h)}
  f(h) = f^* + \alpha h^n,
\end{equation}
where $n$ is the order convergence of the method.

Now suppose that we know the numerical solution for three values of
$h$, $h_1$, $h_2$, and $h_3$, with corresponding numerical solutions
$f_1$, $f_2$, and $f_3$. From this information, can we extract $n$ and
pointwise solutions for $f^*$ and $\alpha$? We begin by devising a
numerical way to extract $n$. 
\begin{eqnarray}
  \frac{f_1 - f_3}{f_2-f_3} &=& \frac{(f^* + \alpha h_1^n) - (f^*+\alpha h_3^n)}{(f^* - \alpha h_2^n) - (f^* + \alpha h_3^n)}\nonumber\\
  \label{eq:n:equation}
  \Rightarrow \frac{f_1 - f_3}{f_2-f_3} &=& \frac{h_1^n - h_3^n}{h_2^n - h_3^n}.
\end{eqnarray}
Although in general it is not analytically solvable, one can use
one-dimensional root-finding to solve equation \eqref{eq:n:equation}
for $n$. Theoretically, one should perform this test at every lattice
point or use curve-fitting methods like least-squares to solve for the
value of $n$. However, it is probably sufficient to solve for $n$ at a
single well-chosen grid point.

Once $n$ is known, we can solve for $\alpha$:
\begin{eqnarray}
  f_1 - f_3 &=& \alpha(h_1^n - h_3^n)\nonumber\\
  \label{eq:alpha:equation}
  \Rightarrow \alpha &=& \frac{f_2 - f_3}{h_2^n - h_3^n}.
\end{eqnarray}
$\alpha$ will very likely depend on $x$.

Finally, we can extract the true solution using $n$, $\alpha$ and any
one of the numerical solutions:
\begin{equation}
  \label{eq:true:solution}
  f^* = f_i - \alpha h_i^n\ \forall\ i\in\{1,2,3\}.
\end{equation}
It's probably best to use the most fine-grained solution one has to
extract the true solution.

\end{document}